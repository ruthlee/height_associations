\documentclass[a4paper,11pt]{article}
\usepackage[T1]{fontenc}
\usepackage[u  tf8]{inputenc}
\usepackage{mathtools}
\usepackage{mathrsfs}
\usepackage{bm}
\usepackage{amsfonts}
\usepackage[dvipsnames]{xcolor}
% \usepackage{cleveref}
\usepackage[normalem]{ulem}
\usepackage{graphicx}
\usepackage{fullpage}
\usepackage[margin=1in]{geometry}
\usepackage{natbib}
\usepackage{amsmath}
%\usepackage{float}
%\usepackage{subcaption}
%\usepackage{multirow} 
\usepackage[colorlinks = true,urlcolor=blue]{hyperref}
%\usepackage{bibunits}
%\usepackage{csvsimple}
%\usepackage[superscript,biblabel]{cite}
\usepackage{verbatim}
\usepackage{graphicx}
\graphicspath{ {images/} }
\usepackage{subfig}
\usepackage{subcaption}

\newcommand{\jb}[1]{{\color{blue} (#1)} }
\begin{document}

\title{\vspace{-2cm}
  Title here
}
\author{Kyelee Ruth Fitts, Jeremy Berg, Guy Amster, Guy Sella}
\maketitle


\subsection*{Abstract}

\subsection*{Introduction}
\jb{Quantifying genetic adaptation is one goal of population geneticists,
  but there are certainly others. It's a little unclear what ``change in variation''
  means. Perhaps it could be better to say something like
  ``changes in the mean phenotype of a population...''.
}

It has long been known that diversity on Earth is in large part due to the
adaptation of organisms to varying environments. We now know that
adaptation has a large genetic basis, and that genetic variation is
the key to biological diversity. Thus, understanding how and why such
genetic variation occurs is the ultimate goal of evolutionary
biology. Quantifying genetic adaptation is the goal of population
geneticists, and to this end finding methods to detect signals of
selection within the genome is of particular interest. Change in
variation over time can be attributed to many causes including genetic
drift or population migration, so detecting signals of selection in
particular is no trivial task. 

\jb{I think this is too early for this paragraph. To start, I would switch this paragraph with the next one. You want to give big picture motivation sort of stuff before you get into any details. }

Wright first introduced the parameter $F_{st}$ as a measure comparing
the variation of an allele in a subpopulation to that of the entire population
\cite{Fst}. Lewontin and Krakauer, using this parameter developed a
novel statistical test based on the fact
that under no selection, the variation of specific alleles over all
subpopulations will not be different from that of the population
$F_{st}$ \cite{firstseltest}. These conclusions were extended by Spitze, who coined the
parameter $Q_{st}$ as a measure of how the phenotypic value of an
allele in a subpopulation compares to that of the entire
population \cite{Qst}. These early tests for selection utilized the ratio
$\frac{Q_{st}}{F_{st}}}$ as a test statistic for detecting signals of
selection, where $F_{st} = Q_{st}$ is the null model, where no
selection occurs. 



Until recently, the study of detecting selection has been limited to
large-effect alleles, when in fact numerous phenotypic traits of interest
are affected by many different loci across the genome. Understanding
how to find signals of selection for such polygenic traits has been
the subject of much study in recent years. Increased compuational
capabilities have given us to tools to be able to evaluate the effects
of many thousands of loci on a trait, allowing us to evaluate
selection in highly polygenic traits, such as height. One of the most
important tools developed are Genome-Wide Association Studies (GWAS)
which determine which loci across the genome have significant effects
on a certain polygenic trait \cite{gwasoverview}. 

\jb{this plunges into details really fast. I think here I would combine some of the content from the paragraph above about $F_{ST}$ and whatnot with what you have below for a more extended introduction of the basic ideas here. I have some more suggestions for how to do this that we can talk about on Tuesday.}

Berg \jb{and Coop} introduces a comprehensive test statistic, called $Q_x$, to test
for selection \cite{berg}. This statistic depends on $F_{st}$ and
a generalized analogy to $Q_{st}$, expressed in terms of $\alpha$, or
the effect size of each significant locus as determined by GWAS, and
the allele frequencies $p$ over loci $l$ and $l'$, summed over
populations $m$. $V_a$ is the additive genetic variance of the entire population.

\begin{equation} \label{Qxraw}
  Q_X = \frac{1}{V_A F_{ST}} \sum_{m=1}^M \sum_{\ell=1}^L \sum_{\ell\prime=1}^L \alpha_{\ell} \alpha_{\ell^{\prime}}\left(p_{m\ell} - \overline{p}_\ell \right)\left(p_{m \ell\prime} - \overline{p}_{\ell\prime}\right)
\end{equation}

\jb{This next bit feels a bit muddled. I think what you mean to say is that alpha is the effect size measured in GWAS, but it sounds like you're saying Qx is used to determine significance in GWAS. Also, I would also probably introduce the idea of dominance and it's relationship to alpha before I introduced Qx}

The distribution of this statistic is expected to be chi-squared. We are particularly interested in the average effect of this
statistic-- not only does GWAS use it to determine which alleles are deemed
significant to a trait, but it is also the weighting factor for loci in the
expression for $Q_x$. The average effect size depends on the allele
frequency, the homozygous effect size (A), or effect size of the most frequent homozygote, and the
dominance deviation (D), which is the difference in effect size for the heterozygote
deviating from $\frac{1}{2}$ of the homozygous effect size.

\begin{equation}  
  \alpha_\ell = \frac{1}{2} A_\ell + D_\ell\left(1-2p_{1\ell}\right).
  \label{alpha}
\end{equation}

The use of $\alpha$ in this test statistic becomes problematic in the
presence of directional dominance, when alleles with a postive effect
size on the trait are systematically dominant and alleles with a
negative effect size on the trait are systematically negative \jb{or vice versa}. In
terms of \eqref{alpha}, this means that the signs of A and D are the
same for a large number of loci \jb{not necessarily the same, just correlated}.

In the presence of directional dominance, we hypothesized a bias in
tests for polygenic adaptation that depends on $\alpha$. Alleles which
are systematically dominant and which have recently increased in
frequency (large p, positive D) will tend to have smaller effect sizes, while
alleles which are recessive which have recently decreased in frequency
(small p, negative D) will tend to have larger effect sizes. In the
latter case, the test for polygenic selection advanced by Berg \jb{and Coop}
\eqref{Qxraw} will tend to make false positive judgements for
selection, because the statistic will be calculated over alleles which
do not actually increase the effect size.


\jb{Again, would move some of this info further up. Can give more suggestions on Tuesday.}
Height is one polygenic trait with many well-defined signficant
alleles via GWAS \cite{heightselection}. It has also been shown to exhibit
directional dominance \cite{heightdirectdom}. We aim to quantify the
hypothesized bias, first with simulated populations, then with height genotype data from the UK Biobank.



\subsection*{Theory/Methods}

Using the expression for the test statistic $Q_x$ \eqref{Qxraw}, we
can substitute the expression for alph \eqref{alpha} and manipulate
the expression algebraically to derive the following expansion for the
test statistic, in terms of the homozygous effect (A)  and the dominance
deviation (D):

\begin{equation}
  \begin{split}
  \sum^L_{l=1}( \frac{1}{2}A_l(p_{1l}-\epsilon_l))^2+\sum^L_{l=1}\sum^L_{
    l \neq l'}(\frac{1}{4}A_l(p_{1l}-\epsilon_{l})A_{l'}(p_{1l'}-\epsilon_{l'}))
  \\
  +\sum^L_{l=1}A_lD_l(1-2p_{1l})(p_{1l}-\epsilon_l)^2 +
  \sum^L_{l=1}\sum^L_{l \neq
    l'}(\frac{1}{2}A_lD_{l'}(1-2_{p1l'})(p_{1l}-\epsilon_l)(p_{1l'}-\epsilon_{l'}) \\
  + \frac{1}{2}D_lA_{l'}(1-2_{p1l})(p_{1l}-\epsilon_l)(p_{1l'}-\epsilon_{l'})) \\
   + \sum^L_{l=1} (D_l(1-2p_l)(p_{1l}-\epsilon_{1}))^2
   + \sum^L_{l=1}\sum^L_{l \neq
     l'}D_{l}(1-2p_{l})(p_{1l}-\epsilon_{1})D_{l'}(1-2p_{l'})(p_{1l'}-\epsilon_{l'}) \label{expansion}
  \end{split}
\end{equation}

We can loosly consider the single summation terms to be variances
corresponding to the expansion of additive effects multiplied by
additive effects, additive times dominance, and dominance times
dominance, and the double summation terms as covariances of these
quantities. We expect the inflation of the test statistic due to
dominance effects to come from the last two terms. Note that when
dominance is not present ($D=0$) the expansion reduces to the
expression Berg presents when $\alpha$ is treated as a constant
\cite{berg}.

Using this expansion, we have created simulations to characterize our
hypothesized dominance bias. We used simulated populations
under a couple of assumptions: first, that the values of the dominance
deviations and homozygous effects are constant throughout the
population. Second, that $F_{st}$ for these simulated approximations
can be roughly estimated by the number of generations elapsed over the
population size. Third, that the distribution of allele frequencies after
one generation can be approximated by a normal distribution centered
at the ancestral frequency with variance
$F_{st}*\epsilon*(1-\epsilon)$ where $\epsilon$ is the ancestral
frequency. (CITE THESE ASSUMPTIONS???)

All simulations were performed in R. 

\subsection*{Results}

\subsection*{Discussion}


\bibliography{works_cited}
\bibliographystyle{plain}






\end{document}
