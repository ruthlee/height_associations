\documentclass[a4paper,10pt]{article}
\usepackage[T1]{fontenc}
\usepackage[u  tf8]{inputenc}
\usepackage{mathtools}
\usepackage{mathrsfs}
\usepackage{bm}
\usepackage{amsfonts}
\usepackage[dvipsnames]{xcolor}
% \usepackage{cleveref}
\usepackage[normalem]{ulem}
\usepackage{graphicx}
\usepackage{fullpage}
\usepackage[margin=0.75in]{geometry}
\usepackage{natbib}
\usepackage{amsmath}
%\usepackage{float}
%\usepackage{subcaption}
%\usepackage{multirow} 
\usepackage[colorlinks = true,urlcolor=blue]{hyperref}
%\usepackage{bibunits}
%\usepackage{csvsimple}
%\usepackage[superscript,biblabel]{cite}
\usepackage{verbatim}
\usepackage{graphicx}
\graphicspath{ {images/} }
\usepackage{subfig}
\usepackage{subcaption}

\newcommand{\jb}[1]{{\color{blue} (#1)} }
\begin{document}


\subsection*{Project Progress Report: W3500 Independent Biological Research}
\subsubsection*{Kyelee Ruth Fitts}

The (chi-squared distributed) test statistic $Q_x$ is used in population genetics to determine
whether alleles with significant effect sizes (usually determined by
GWAS) contribute to selection on a polygenic trait. Specifically,
given a certain number of populations, it measures whether the effect
sizes of each locus within each population show a trend of selection
for or against the polygenic trait by measuring whether the covariance
of genetic value (allele frequency minus mean ancestral frequency) is
nonzero over all loci with significant effect sizes.

$Q_x$ depends on the value of the average effect size at each
locus, which in turn depends on the allele frequency, the homozygous
effect size (effect size of the most frequent homozygote) and the
dominance deviation (difference in effect size for the heterozygote
deviating from $\frac{1}{2}$ of the homozygous effect size).

Directional dominance is when heterozygotes display greater
effect sizes than expected-- that is, the signs of
the dominance deviation and homozygous effect sizes are the same. In
the presence of directional dominance, we hypothesize that there is a
bias in the $Q_x$ statistic: for alleles displaying directional
dominance, alleles which have recently increased in frequency
(dominant alleles)  will tend to display smaller effect sizes, and alleles which have recently decreased in
effect sizes (recessive alleles) will tend to display larger effect
sizes. In this latter case, false positives for polygenic selection
will occur. Below is a theoretical justification for this hypothesis.

$Q_x$ can be
expressed in terms of the effect size $\alpha$ in this relationship \cite{gwasintro}:

\begin{equation} \label{eqn:Qxraw}
  Q_X = \frac{1}{V_A F_{ST}} \sum_{m=1}^M \sum_{\ell=1}^L \sum_{\ell\prime=1}^L \alpha_{\ell} \alpha_{\ell^{\prime}}\left(p_{m\ell} - \overline{p}_\ell \right)\left(p_{m \ell\prime} - \overline{p}_{\ell\prime}\right)
\end{equation}

Where $\alpha$ can be represented in terms of the dominance deviation
$D_l$ at each locus $l$ and
the homozygous effect $A_l$ at each
locus:

\begin{equation}  \label{eqn:avgeff}
  \alpha_\ell = \frac{1}{2} A_\ell + D_\ell\left(1-2p_{1\ell}\right).
\end{equation}

Substituting \eqref{eqn:avgeff} into \eqref{eqn:Qxraw}, we can
algebraically derive the following expansion:

\begin{equation}
  \begin{split}
  \sum^L_{l=1}( \frac{1}{2}A_l(p_{1l}-\epsilon_l))^2+\sum^L_{l=1}\sum^L_{
    l \neq l'}(\frac{1}{4}A_l(p_{1l}-\epsilon_{l})A_{l'}(p_{1l'}-\epsilon_{l'}))
  \\
  +\sum^L_{l=1}A_lD_l(1-2p_{1l})(p_{1l}-\epsilon_l)^2 +
  \sum^L_{l=1}\sum^L_{l \neq
    l'}(\frac{1}{2}A_lD_{l'}(1-2_{p1l'})(p_{1l}-\epsilon_l)(p_{1l'}-\epsilon_{l'}) \\
  + \frac{1}{2}D_lA_{l'}(1-2_{p1l})(p_{1l}-\epsilon_l)(p_{1l'}-\epsilon_{l'})) \\
   + \sum^L_{l=1} (D_l(1-2p_l)(p_{1l}-\epsilon_{1}))^2
   + \sum^L_{l=1}\sum^L_{l \neq
     l'}D_{l}(1-2p_{l})(p_{1l}-\epsilon_{1})D_{l'}(1-2p_{l'})(p_{1l'}-\epsilon_{l'}) \label{second_exp_ult}
  \end{split}
\end{equation}

Where we can expect much of the bias in the $Qx$ statistic coming from
dominance to come from the last two terms involving $D_l$. Since the
semester has begun, I've been creating simulations to verify the
effect that dominance has on the test statistic. On the second page,
Figure 1 shows the distribution of the $Q_x$ statistic without dominance as simulated by
the expansion above, which matches a chi-squared distribution with
degree of freedom equal to 1, as expected.

Figures 2, 3, and 4 compare the expected cumulative distribution
function of the $Q_x$ statistic (in red) with the cdf of the simulated
distribution. When the dominance deviation is zero, as expected the
two are nearly identical. However, as dominance deviation increases,
the distribution of the $Q_x$ statistic shifts further and further to
the right. This is further shown in figure 5, which plots the fraction
of the simulated statistics above the value of the statistic expected
for 5\% of the expected distribution, which is chi-squared with 1
df. That is, the plot shows the proportion of $Q_x$ values above the p = 0.05 cutoff value
of the expected chi-squared distribution for different dominance
deviation values. As expected, this plot also
shows that the proportion of simulated $Q_x$ values over the expected
threshold increases as the effect of dominance increases.

Finally, figures 6, 7, and 8 show how dominance affects the genetic
value (that is, the change in allele frequency multiplied by the effect size $\alpha$)
over time. These figures are especially interesting because they show
how specifically directional dominance can affect genetic value. When the
dominance deviation is equal to 0 in figure 6, as expected the genetic
values of the replicates drift randomly over time and the mean line
(red) is approximately zero. However, for positive dominance
deviations (fig. 7), the genetic value over time becomes negative, while for
negative dominance deviations, the genetic value over time
becomes positive (compare with the blue line at y=0). This indicates, crucially, that the bias due to dominance will
increase the perceived effect size and result in false positives if alleles tend to be recessive 
(as with height) and decrease the perceived effect size for alleles
displaying dominance, as hypothesized.

So far the work I've been doing this semester has been mainly in
assembling these simulations to directly quantify the statistical bias
in these tests due to dominance. I am currently working on a simulation that will break
down the terms of the expansion to see what effect each term has
separately on the distribution of the statistic. In addition, I have just gained access to the
UK BioBank data in order to search for the bias within human genetic
data. 

\bibliography{works_cited}
\bibliographystyle{plain}

\pagebreak

\begin{figure}
  \caption{Distribution of Qx Expansion (Chi-squared, df=1)}
  \centering
  \includegraphics[width=0.5\textwidth]{domdevdist}
\end{figure}

\begin{figure}
  \caption{CDF of Qx Statistic vs expected cdf where dominance
    deviation = 0}
  \centering
  \includegraphics[width=0.5\textwidth]{cdfdomdev=0}
\end{figure}

\begin{figure}
  \caption{CDF of Qx Statistic vs expected cdf where dominance
    deviation = 0.5}
  \centering
  \includegraphics[width=0.5\textwidth]{cdfdomdev=05}
\end{figure}

\begin{figure}
  \caption{CDF of Qx Statistic vs expected cdf where dominance
    deviation = 1}
  \centering
  \includegraphics[width=0.5\textwidth]{cdfdomdev=1}
\end{figure}

\begin{figure}
  \caption{Genetic value of locus vs time, dominance deviation = 0}
  \centering
  \includegraphics[width=0.5\textwidth]{genvalvstime}
\end{figure}

\begin{figure}
  \caption{Genetic value of locus vs time, dominance deviation = 0.2}
  \centering
  \includegraphics[width=0.5\textwidth]{genvalvstime02}
\end{figure}

\begin{figure}
  \caption{Genetic value of locus vs time, dominance deviation = -0.2}
  \centering
  \includegraphics[width=0.5\textwidth]{genvalvstime-02}
\end{figure}


\end{document}