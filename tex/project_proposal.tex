\documentclass[a4paper,10pt]{article}
\usepackage[T1]{fontenc}
\usepackage[u  tf8]{inputenc}
\usepackage{mathtools}
\usepackage{mathrsfs}
\usepackage{bm}
\usepackage{amsfonts}
\usepackage[dvipsnames]{xcolor}
% \usepackage{cleveref}
\usepackage[normalem]{ulem}
\usepackage{graphicx}
\usepackage{fullpage}
\usepackage[margin=0.75in]{geometry}
\usepackage{natbib}
\usepackage{amsmath}
%\usepackage{float}
%\usepackage{subcaption}
%\usepackage{multirow} 
\usepackage[colorlinks = true,urlcolor=blue]{hyperref}
%\usepackage{bibunits}
%\usepackage{csvsimple}
%\usepackage[superscript,biblabel]{cite}
\usepackage{verbatim}

\newcommand{\jb}[1]{{\color{blue} (#1)} }
\begin{document}

\title{\vspace{-2.0cm}
  Project Proposal \\
  \large W3500 Independent Biological Research \\
}

\author{
  Kyelee Ruth Fitts
}

\date{Fall 2017}
  
%\maketitle

\subsection*{Project Proposal: W3500 Independent Biological Research}
\subsubsection*{Kyelee Ruth Fitts}

% I think you want to start a bit broader than this. Adaptation
% is a major factor in generating the diversity of life across the planet,
% and studying the process of adaptation gives us some insight into how that diversity is generated.
% Part of this entails understanding the genetic basis of adaptation. Until recently, however,
% we've been limited almost entirely to studying adaptations consisting of individuals mutations of
% of large effect, when in fact we know that many traits are highly polygenic, and thus adaptations
% involving these traits would not show signals of adaptation in tests designed to detect large effect alleles.
%
% this paper might be useful as a review of adaptations by large effect alleles in human populations:
% http://science.sciencemag.org/content/354/6308/54
% the intro to my 2014 paper also has some example verbiage about this large effect alleles vs polygenic adaptation divide
%
%
% An alternative tack to motivate things is the one used in that paper. As humans have spread around
% the globe over the last 200,000 years they have adjusted live in a range of new habitats, adapting
% via a combination of biological and cultural mechanisms. Population genetic tools have been useful
% for identifying large effect mutations involved in biological adaptations, but we know that many
% traits are highly polygenic, and therefore these previous approaches provide a biased picture of
% human adpatation.
%
% From here, you can transition into your current material about GWAS etc.

Diversity of life on Earth is in large part due to the adaptation of living
organisms to the circumstances of their environments. As adaptation is
a fundamentally genetic process, understanding the genetic mechanisms
by which populations adapt to their environments is a fundamental goal
of evolutionary biology. Until recently, the study of the genetics
underlying adaptation has been focused on individual mutations of large
effect size, when in fact we know that many important traits of interest are polygenic traits that are affected by many
different loci.

In order to study such polygenic traits, genome-wide association studies (GWAS) have become increasingly
important in the field of population genetics. Utilizing advanced
genotyping technology and very large sample sizes, GWAS provides
estimates for the extent of correlation between traits and loci as well as the
effect sizes of significant alleles over the entire genome. GWAS in combination with statistically rigorous tests of
polygenic adaptation have been used to detect significant and
systematic correlations between the changes of allele frequencies at
each site and the effect that each allele has on the trait in
question-- in effect, finding a footprint of natural selection
against the noise of genetic drift. These tests for polygenic
adaptation using GWAS data have been used to detect signals of
selection in many anthropometric traits like height
and waist-hip ratio, as well as disease phenotypes like type 2
diabetes\cite{gwasintro,gwasproblems}.

As these tests for polygenic adaptation become more and more ubiquitous, it is necessary to ensure that the underlying
assumptions of the statistical tests used remain as free from bias as
possible to prevent false signals of selection. 

One important model for studying polygenic traits is the additive model, where

\begin{equation}
  y = \mu + \alpha g + \epsilon
\end{equation}

This model refers to the trait value, $y$, of each SNP in an
individual. $\mu$ is the average phenotype of the population. $g$ refers to the
allelic dosage (i.e. 0, 1, or 2 copies of the allele), $\alpha$ is the average effect size
of each allele on the phenotype, and $ \epsilon $ is a residual term which captures both the
effects of the environment and all other loci. GWAS use robust statistical methods
to assess the evidence that an allele has a
significant effect on the value of a certain quantitative
trait, as well as to estimate the size of this effect.

\begin{comment}
  Ok, this is getting close, but I think there are still a couple things to unwind here.
  The second equation below is not really a feature of tests of polygenic adaptation, but rather is just a fact about the way that the average effect size depends on the difference between homozygotes, dominance, and the current allele frequency in the population. Above, you've already motivated the fact that tests of polygenic adaptation assume independence of the average effect and patterns of change in allele frequency. Now I think before you get to equation 2 you want to introduce the idea of dominance (you haven't done that up to this point), i.e. when the effect of an allele on a phentoype depends on the identity of the other allele at that site, and then you can say that the presence of dominance violates this assumption of independence between allele frequency and average effect. Then show the equation and explain what the individual pieces are.
  \end{comment}


Dominance is well understood in population genetics to mean that the
effect an allele has on an organism's phenotype depends on the other
allele at the locus. For instance, if a dominant (in the traditional
sense) allele is represented by $A$ and a recessive allele by $a$,
then, the effect size of the allele $a$ depends on whether or not
there is an $A$ at the site. Polygenic traits that exhibit dominance
pose a problem for the assumption underlying tests of polygenic
adaptation: that allele frequency and average effect size
are uncorrelated. 

A well-defined equation for average effect size is: 

\begin{equation}
  \alpha = a - a(2h - 1)(2q - 1)
\end{equation}

Where $a$ is half the phenotypic difference between homozygotes, $h$ is the
dominance coefficient, indicating how the phenotype of the heterozygote deviates from halfway between the homozygotes, and $q$ is the frequency of the allele for which
the effect size is being measured. Tests for polygenic adaptation
assume that the average effect an allele has on a trait ($\alpha$) is
uncorrelated with the change in allele frequencies over time
($q$). However, in the presence of dominance, this assumption is violated in two ways, because 1)
the allele frequency used in the model comes from the population on
which the GWAS was performed and 2) because through directional
dominance, $a$ and $h$ can be correlated. Specifically, if alleles that
increase the effect size of a trait also tend to be recessive
($h<\frac{1}{2}$), then a GWAS is more likely to identify alleles that have
recently increased in frequency (larger $q$) as significant due to
their seemingly larger average $\alpha$, leading to false
positive signals of polygenic adaptation.

\begin{comment}
  
\jb{Somewhere around here, I think you could add a second equation showing how the average effect depends on the difference between
  homozygotes and the dominance coefficient. This is the equation which appears on slides 8, 9, and 10 of Yuval's powerpoint.
  Tests of polygenic adaptation generally implicitly make an assumption that $h$ (the dominance coefficient) is equal to 1/2
  (i.e. that there is no additivity; note that when $h=1/2$ the second term in that equation is $0$, and the average effect
  of an allele is independent of its frequency in the population). So you could insert the equation
  \begin{equation}
    \alpha = a - a(2h-1)(2q-1)
  \end{equation}
  where a is half the difference in phenotype between homozygotes, $h$ is the dominance coefficient, and $q$ is the frequency of
  the allele in the population where the effect size is estimated. The issue for polygenic adaptation tests is that if the effect
  size is estimated in a population that is also included in the polygenic adaptation test (which is often unavoidable), AND there
  is directional dominance (i.e. $a$ and $h$ are correlated, and therefore $a(2h-1)$ tends to be systematically greater than or
  less than zero, depending on the direction of dominance, i.e. whether h is usually less than 1/2 or greater than 1/2 for trait
  increasing alleles), THEN the average effect ($\alpha$) will tend to be larger at sites where the allele has recently increased
  (or decreased, again depending on the direction of dominance) in frequency. This violates the assumption of independence between
  effect size and direction of allele frequency change under neutrality in  tests of polygenic adaptation, and is why directional
  dominance can potentially generate false signals in such tests adaptation. This is more than you want to explain, but perhaps you
  could include this second equation (which would more strongly justify showing the first I think) and try to give a brief summary.
  I've rearranged your test below a little bit as a first stab at expressing what I think we're trying to say here.}

\jb{However, polygenic adaptation tests assume as a null model that the
average affect an allele has on a trait ($\alpha$) is uncorrelated with patterns of allele frequency change over time, a condition
which is violated when directional dominance is present. Specifically, if alleles that increase the effect
size of a trait tend to be recessive, then GWAS will be more likely to identify alleles which have recently increased in frequency
(due to their larger average effect size estimates relative to those which have decreased in frequency), and this may generate false positive
signals in polygenic adaptation tests.}
\jb{I swapped recessive in for dominant in the above paragraph as it more closely matches what the evidence suggests for height}

\end{comment}

To investigate this bias, I plan to work with Dr. Jeremy Berg in the
Sella Lab, approaching this problem in two steps: first, to derive
mathematically an expression that can quantify the bias due to
directional dominance in GWAS given known expressions and concepts in
population genetics. The second step would be to use the expression
derived in step one to measure this bias using real data.

Height is an anthropometric trait for which many studies have found signals
of selection using
evidence from GWAS \cite{heightselection} . However, other studies have shown that
height is also subject to directional dominance \cite{heightdirectdom}
-- a combination that
makes the trait well-suited for the purposes of my research. Data will
come from the recent UK Biobank study, which has gathered genetic data
on about 500,000 participants from the UK \cite{biobank} .

Some progress has already been made on this project. In the spring of
2017 we found using the UK Biobank data further evidence of
directional dominance in height. Over the summer, I worked
with Dr. Berg to begin developing a mathematical expression for the
bias. I hope this semester to make significant progress on what I
believe is a fascinating project in mathematics and biology. 

\bibliography{works_cited}
\bibliographystyle{plain}

\end{document}



