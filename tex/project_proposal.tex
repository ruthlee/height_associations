\documentclass[a4paper,10pt]{article}
\usepackage[T1]{fontenc}
\usepackage[u  tf8]{inputenc}
\usepackage{mathtools}
\usepackage{mathrsfs}
\usepackage{bm}
\usepackage{amsfonts}
\usepackage[dvipsnames]{xcolor}
%\usepackage{cleveref}
\usepackage{graphicx}
\usepackage{fullpage}
\usepackage{natbib}
\usepackage{amsmath}
%\usepackage{float}
%\usepackage{subcaption}
%\usepackage{multirow} 
\usepackage[colorlinks = true,urlcolor=blue]{hyperref}
%\usepackage{bibunits}
%\usepackage{csvsimple}
%\usepackage[superscript,biblabel]{cite}

\newcommand{\jb}[1]{{\color{blue} (#1)} }
\begin{document}

\title{\vspace{-2.0cm}
  Project Proposal \\
  \large W3500 Independent Biological Research \\
}

\author{
  Kyelee Ruth Fitts
}

\date{Fall 2017}
  
\maketitle


GWAS (Genome-Wide Association Studies) have become increasingly
important in the field of population genetics as a method of detecting
polygenic adaptation, or adaption via traits that are affected by many
different loci instead of just one as per traditional Mendelian
genetics. GWAS have been used to detect polygenic adaptation in
disease traits, for example detecting signals of selection in Type 2
diabetes, as well as a whole host of other quantitative traits
that give valuable insight into how adapted genetic differences
can be found, studied, and usefully applied \cite{gwasintro}
\cite{gwasproblems} . As this method of
analysis becomes more and more crucial to analyzing such
anthropometric traits, it is necessary to ensure that the underlying
assumptions of GWAS and its methods remain as free from bias as
possible to prevent false signals of selection. 

One important model for studying polygenic traits is the additive model, where

\begin{equation}
  y = \mu + A\alpha
\end{equation}

This model refers to the trait value, $y$, of each SNP in an
individual. $\mu$ is the average phenotype of the population. $A$ refers to the
parametrization of the genotypes and $\alpha$ is the average effect size
of each allele on the phenotype. GWAS use robust statistical methods
to obtain p-values that indicate whether a certain allele has a
significant affect on the expression of a certain quantitative
trait. However, GWAS assume that the
average affect an allele has on a trait ($\alpha$) is uncorrelated with
the allele frequency over time of the trait-- that is, uncorrelated
with any dominance effects of the trait which could cause the allele
frequency to shift over time or over different populations.

However, we know that under directional dominance, that assumption is
not necessarily true. That is, if alleles that increase the effect
size of a trait are dominant, then that allele will have a greater
frequency and GWAS will overestimate the effect size-- a bias that has
the potential to result in false positive results.

To investigate this bias, I plan to work with Dr. Jeremy Berg in the
Sella Lab, approaching this problem in two steps: first, to derive
mathematically an expression that can quantify the bias due to
directional dominance in GWAS given known expressions and concepts in
population genetics. The second step would be to use the expression
derived in step one to measure this bias using real data.

Height is an anthropometric trait for which many studies have found signals
of selection using
evidence from GWAS \cite{heightselection} . However, other studies have shown that
height is also subject to directional dominance \cite{heightdirectdom}
-- a combination that
makes the trait well-suited for the purposes of my research. Data will
come from the recent UK Biobank study, which has gathered genetic data
on about 500,000 participants from the UK \cite{biobank} .

Some progress has already been made on this project. In the spring of
2017 we found using the UK Biobank data further evidence of
directional dominance in height. Over the summer, I worked
with Dr. Berg to begin developing a mathematical expression for the
bias. I hope this semester to make significant progress on what I
believe is a fascinating project in mathematics and biology. 

\bibliography{works_cited}
\bibliographystyle{plain}

\end{document}



